\chapter{Supporting Materials}

This would be any supporting material not central to the dissertation.
For example:
\begin{itemize}
\item additional details of methodology and/or data;
\item diagrams of specialized equipment developed.;
\item copies of questionnaires and survey instruments.
\end{itemize}

\begin{table}[]
	\caption{List of hyperparameters used for experiments }
	\label{tab:hyperparams}
	\center
	\begin{tabular}{|c|c|}
		\hline
		\multicolumn{2}{|c|}{\textbf{Optimization Parameters}} \\ \hline
		Optimizer                        & Adam                \\ \hline
		Learning Rate                    & 0.0003              \\ \hline
		KL Annealing Schedule            & 80,000 steps        \\ \hline
		Word Dropout Rate                & 0.1                 \\ \hline
		Clip nNorm                        & 1.0                 \\ \hline
		Mini Batch Size                  & 64                  \\ \hline
		Number of Samples (ELBO)         & 10                  \\ \hline
		\multicolumn{2}{|c|}{Model Parameters}                 \\ \hline
		Source Embedding Size            & 256                 \\ \hline
		Target Embedding Size            & 256                 \\ \hline
		Encoder Hidden Dimensions        & 256                 \\ \hline
		Number of Encoder Layers         & 1                   \\ \hline
		Decoder Hidden Dimensions        & 256                 \\ \hline
		Number of Decoder Layers         & 1                   \\ \hline
		Dropout                          & 0.5                 \\ \hline
		Z dim (latent variable)          & 2 , 218, 256        \\ \hline
		\multicolumn{2}{|c|}{Global Attention Mechanism}       \\ \hline
		Key Size                         & 512                 \\ \hline
		Query Size                       & 256                 \\ \hline
		\multicolumn{2}{|c|}{I.A.F Details}                    \\ \hline
		Autoregressive NN                & 320 Units           \\ \hline
		\multicolumn{2}{|c|}{Planar Flows Details}             \\ \hline
		Hidden layer dimensions          & 150                 \\ \hline
	\end{tabular}
\end{table}