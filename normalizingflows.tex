\chapter{Normalizing Flows in Machine Translation}

%In this chapter we discuss the implementation details for the \ac{LVNMT} models we consider for analysis. Along with this includes several design choices we made based on related research and our own findings in the preliminary experiments section. 

In this chapter we discuss our approach to incorporating normalizing flows into the \ac{LVNMT} systems discussed in Chapter 3. This includes descriptions of the normalizing flows we considered, and regularization techniques to improve the utilization of the latent variable with auto-regressive models.
\section{Applying Flows to Latent Variables}


As a reminder, normalizing flows transform samples from a base distribution $p(z_{0})$ to samples of more complex distribution by applying $k$ invertible functions $f_{i}$ sequentially: 

\begin{equation}
z_{k} = f_{k} \circ f_{k-1} ... \circ f_{2} \circ f_{1}(z_{0}) , z_{0} \sim p(z_{0})
\end{equation}

In our \ac{LVNMT} models, $p(z_{0})$ refers to our variational posterior distributions. Each $f_{i}$ can be viewed as adding additional network layer between the base Gaussian distribution and the designated part of the translation model that the latent variable Z is included as input (see figure).

We follow previous research which make flows data dependent \cite{rezende2015VIwithNF,Berg2018SylvesterNF,kingma2016IAF,tomczak2016Householder}.  Generally speaking, this means each input sentence $x$ will have unique transforms $f_{i}$ enabling more flexible latent distributions per sentence pair. We leave further details on this to the next section as each flow handles this differently. Agnostic to the flow choice, we condition on the source sentence via $h_{mean}^{X}$. \footnote{As a reminder, this is the averaged hidden representation of the source sentence produced by our inference network.} This choice was because in either model we can only condition on $x$ at decode time.

During training, we optimize the \ac{ELBO}, which we present again as formulated more specifically to machine translation case for the discriminative model, and is based on the derivation from \citet{rezende2015VIwithNF}, Section 4.2:
\begin{align}
\begin{split}\label{eq:2}
&
E_{q(\textbf{z}_{0}\cond{\textbf{x}, \textbf{y}})} \bigg[ \sum_{j=1}^{p} \log p_{\theta}(y_{j} \cond{\textbf{z}_{k}}, \textbf{x}, y_{<j}) \bigg] \\
& - KL(q_{\phi}(\textbf{z}_{0} \cond{\textbf{x}, \textbf{y}}) || p_{\theta}(\textbf{z}_{k} \cond{\textbf{x}})) \\
&   +  E_{q_{\phi}(\textbf{z}_{0} \cond{\textbf{x}, \textbf{y}})} \bigg[\sum_{k=1}^{K} \log \bigg|\frac{\delta f_{k}}{\delta \textbf{z}_{k-1}} \bigg| \bigg].
\end{split}
\end{align}
This simply introduces maximizing the sum of  log absolute Jacobian from our flows. \footnote{In the joint distribution case, there would be the inclusion of optimizing a language model as well.}

Unfortunately, in normalizing flows we cannot analytically derive the KL divergence and instead we must perform Monte Carlo sampling to optimize this objective. When evaluating translation quality, we set the latent variable to the expected value of the Gaussian distribution , $z_{0} = \mu_{\theta}(x)$, and apply our flows to this value.

Specific to our discriminative \ac{LVNMT} system, we choose to share the flow parameters on both the prior and variational distribution. At decode time, we replace the variational distribution with the prior which as been optimized to generate distributions similar to the variational posterior. Learning separate flows for each distribution would otherwise be unnecessary computation overhead as the two base distributions should ideally match each other. This also changes the above equation to be quite similar to the original \ac{ELBO} without normalizing flows, with the exception that we use $z_{k}$ instead of $z_{0}$.


%\reminder{ move talking about objective in this section as more general case}





%\reminder{talk more about what is ammortization and what it looks like in your models. there's lots of ways to do this}

%\reminder{ what does it mean}

%\reminder{ We choose to condition flows on only source sentence X}

%\reminder{ for discriminative model where we learn both prior and variational distribution, we only condition on X and apply flows to both models. the rational is that the flows will always be used in either case and would ideally be perfect matches of each other}


\section{Considered Flows for Analysis}

For our analysis we consider two types of flows from the literature, but note that there are a variety of choices as research, at the time of writing, is quite active. The Jacobian is different for each flow we consider, offering alternative influences on the training procedure for our \ac{LVNMT} architectures. 

\subsection{Planar Flows}

Planar flows were proposed in the work of \citet{rezende2015VIwithNF}. They can be viewed as a scale-shift operation of the following form:
\begin{equation}
f_{i}(z) = z + u_{i} h(w_{i}^{T} z + b_{i}).
\end{equation}
Here, $u_{i}, w_{i} \in R^{d}$ and $b \in R$ are the parameters of planar flow $i$, and $h$ is a non-linear activation. For our experiments we use \textit{tanh} but the authors note that alternative activations are permissible. A convenient aspect of these flows is they provide an analytical term of the Jacobian:
\begin{equation}
\bigg| \det \bigg( \frac{\delta f}{\delta z} \bigg)\bigg|  = \bigg| 1 + u^{t} h'(w_{i}^{T} z + b_{i})w \bigg|
\end{equation} 
Here, $h'$ is the derivative of the nonlinear activation. For clarity, the second term in the right hand side of the equation is a dot product. Intuitively, this transformation can be seen as contracting or expanding the samples $z$ along a single plane in space. This is a simple transformation and requires many planar flows to represent complicated distributions.  %This need for more flows to effectively transform the distribution space makes them less computationally efficient compared to other flows, but offer a simple type of normalizing flow to evaluated against.

%Another drawback with planar flows is that their inverse is not guaranteed. \citet{rezende2015VIwithNF} propose a way to define the parameters such that the flows are invertible, but it still remains challenging to invert planar flows. This can be due to numerical instability in the activation function, which when inverted can lead to a non-existent value, or require numerical solvers to find a solution. Despite these potential pitfalls, they are a fairly simple transformation to compare the importance of more complex distributions for LVNMT

%Previous research suggest these data conditioned flows provide better performance compared to data agnostic flows in which the parameters are treated as directly learned parameters \cite{vdberg2018sylvester}.

As previously mentioned, the parameters of flows are generally made to be data dependent. In the case of planar flows this is achieved by utilizing a \textit{hyper-network} \cite{ha2016hypernets} which outputs the parameters of the flow:
 \begin{equation}
 MLP(h^{mean}_{x}) = (u, w, b).
 \end{equation}
In our experiments each of our flows has a separate hyper-network with a single hidden layer with \textit{tanh} activations to enable flows to have sufficient flexibility to transform distributions.
 
\subsection{Inverse Autoregressive Flows}
%\reminder{re-read details on IAF...this is missing a few crucial things..like what the heck is sigma or mu}
Inverse autoregessive flows were proposed by \citet{kingma2016IAF} to enable parallel sampling by defining an invertible function as a sequentially dependent  inverse scale and shift  operation in a sequence of random variables
\begin{equation}
z_{k+1}^{i} = \frac{z_{k}^{i} - \sigma(z_{k}^{1:i-1}, h)}{\mu(z_{k}^{1:i-1}, h)}.
\end{equation}
Here, $z^{i}$ is the dimension $i$ of the vector of the latent variable $z$, $\sigma(\cdot)$ and $\mu(\cdot)$ are the outputs of a \textit{MADE} network \cite{MADE2015germain}. Here, $h$ is referred to as the context input which we use $h= h^{mean}_{x}$. It represents the data conditioning of the normalizing flow, and otherwise the parameters of the MADE layers are shared between data points. Defining a normalizing flow in this way provides a lower triangular Jacobian which means the absolute Jacobian is a product of the diagonal terms 
\begin{equation}
\bigg| \det \frac{\delta f}{\delta z} \bigg| = \prod_{t=0}^{T} \sigma(z_{k}^{1:t-1}, h^{mean}_{x})
\end{equation}

%The inverse is then defined as sequentially dependent which make these flows computationally expensive for density estimation. 

%\begin{equation}
%y_{i} = \mu + \sigma * \epsilon_{<i}
%\end{equation}

%In the authors work they use autoregressive MADE \cite{gregor2015MADE} and generate a context $h$ which each flow conditions on in addition to other inputs. This additional context vector is how each flow is conditioned on the data, as otherwise the MADE layers parameters are shared between each datum. In our models, the context vector is the $h_{mean}^{x}$.







\section{Regularization Tricks}



%fyi
%https://tex.stackexchange.com/questions/44450/how-to-align-a-set-of-multiline-equations

An often cited challenge including latent variables in auto-regressive models is  \textit{posterior collapse} \cite{he2018lagging}. In order to maximize the \ac{ELBO}, the variational distribution parameters, for all the training data, are pushed to more closely match the prior. This leads to uninformative latent variables typically when choosing the isotropic Gaussian as the prior. Part of this behaviour has been accredit to strong decoders, like auto-regressive models, which are flexible enough to model the output even by ignoring $z$. We recommend \citet{chen2016VariationalLossyAE} or \citet{zhao2017InfoVAE} which provide more thorough discussions on the subject. For this work, we address this potential problem with previously proposed approaches referred to as KL-annealing \cite{bowman2015GeneratingSent,sonderby2016LadderVAE} and word dropout \cite{bowman2015GeneratingSent}.

KL-annealing typically involves annealing the weight $\beta$ of the divergence term in the \ac{ELBO}:
\begin{align}
\begin{split}\label{eq:2}
&
E_{q(\textbf{z}_{0}\cond{\textbf{x}, \textbf{y}})} \bigg[ \sum_{j=1}^{p} \log p_{\theta}(y_{j} \cond{\textbf{z}_{k}}, \textbf{x}, y_{<j}) \bigg] \\
& - \beta KL(q_{\phi}(\textbf{z}_{0} \cond{\textbf{x}, \textbf{y}}) || p_{\theta}(\textbf{z}_{k} \cond{\textbf{x}})) \\
&   +  \beta E_{q_{\phi}(\textbf{z}_{0} \cond{\textbf{x}, \textbf{y}})} \bigg[\sum_{k=1}^{K} \log \bigg|\frac{\delta f_{k}}{\delta \textbf{z}_{k-1}} \bigg| \bigg].  
\end{split}
\end{align}
In normalizing flows research, it is often cited that KL-annealing helps improve performance even without strong decoders \cite{rezende2015VIwithNF,kingma2016IAF,tomczak2016Householder,Berg2018SylvesterNF,ziegler2019LatentNFforDiscrete}. In our experiments, we use a linear schedule which increases the importance of our regularization terms after each mini-batch update. 


Word dropout is a procedure during training time where with some probability $\rho$ the current word embedding for $x_{i}$ is replaced with the word embedding for the unknown token.\footnote{The unknown token is used to handle words not included in the vocabulary during training.} The intuition for it's effectiveness is it encourages the model to depend more on the latent variable for information at decoding. This approach has particularly been important for improving performance of generative \ac{LVNMT} models \cite{harshil2018GNMT,eikema2018AEVNMT}.

