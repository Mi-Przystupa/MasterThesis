\chapter{Normalizing Flows in Machine Translation}

In this chapter we discuss the implementation details for the \ac{LVNMT} models we consider for analysis. Along with this includes several design choices we made based on related research and our own findings in the preliminary experiments section. 


In this section we discuss incorporating normalizing flows into latent variable neural machine translation. We also discuss some challenges with incorporating normalizing flows based on the effect they have on the ELBO. 

\subsection{Applying Flows to Latent Variables}
It is relatively easy to incorporate normalizing flows into existing LVNMT models. During the training procedure, one only needs to apply $k$ functions $f_{i}$ sequentially to samples from the base distribution $p(z_{0})$: 

\begin{equation}
z_{k} = f_{k} \circ f_{k-1} ... \circ f_{2} \circ f_{1}(z_{0}) , z_{0} \sim p(z_{0})
\end{equation}

Here, $\circ$ is a shorthand for nested calls of the functions ($f_{2}(f_{1}(f_{0}(z_{0})))$). For our experiments, the base distribution $p(z_{0})$ refers to our variational distribution $q_{\phi}(\textbf{z} \cond{\textbf{x}, \textbf{y}})$ and at decode time we use $p_{\theta}(\textbf{z} \cond{\textbf{x}})$ which is the same approach taken by \citet{Zhang2016VNMT}. For deterministic decoding, we set $z_{0} = \mu_{\theta}(x)$ where $\mu_{\theta}$ is produced by our parameterized prior distribution, and apply normalizing flows on this fixed value instead of samples from $p_{\theta}(z_{0} \cond{x})$

\subsection{Challenges with Optimization}

The one other consideration with the inclusion of normalizing flows is how they change the ELBO. Here we present a formulation of the ELBO specific to machine translation which is based on the derivation from \citealp[Section 4.2]{rezende2015VIwithNF}.

\begin{align}
\begin{split}\label{eq:2}
&
E_{q(\textbf{z}_{0}\cond{\textbf{x}, \textbf{y}})} \bigg[ \sum_{j=1}^{p} \log p_{\theta}(y_{j} \cond{\textbf{z}^{(k)}}, \textbf{x}, y_{<j}) \bigg] \\
& - KL(q_{\phi}(\textbf{z}_{(0)} \cond{\textbf{x}, \textbf{y}}) || p_{\theta}(\textbf{z}^{(k)} \cond{\textbf{x}})) \\
&   +  E_{q_{\phi}(\textbf{z}_{0} \cond{\textbf{x}, \textbf{y}})} \bigg[\sum_{k=1}^{K} \log \bigg|\frac{\delta f^{(k)}}{\delta \textbf{z}^{(k-1)}} \bigg| \bigg]  
\end{split}
\end{align}

The first term represent maximizing the likelihood of observed sequences i.e. translating data correctly. The other two terms represent the introduced regularization from the latent variable $\textbf{z}$ in the model. 
%fyi
%https://tex.stackexchange.com/questions/44450/how-to-align-a-set-of-multiline-equations

The problem with this objective is that the inclusion of this KL divergence can lead to a problem referred to as ''posterior collapse'' \cite{he2018lagging}. This refers to the scenario where, in order to maximize the ELBO, the variational distribution parameters, for all the training data, are pushed to more closely match the prior distribution parameters. In the typical case where the prior is the unit Gaussian distribution, this leads to uninformative codes in which case the latent variable provides no additional information to the model. We recommend \citet{chen2016VariationalLossyAE} or \citet{zhao2017InfoVAE} which provide more thorough discussions on the subject. 

 

For this work, we address this potential problem with a previously proposed approach referred to as KL-annealing \cite{bowman2015GeneratingSent,sonderby2016LadderVAE}. KL-annealing is the process of annealing the weight associated to the divergence term in the ELBO from 0.0 (no influence) to 1.0 (original weight). We follow previous research by using a linear annealing schedule to update the weight of our regularization terms after each mini-batch update until it reaches 1.0 and the original ELBO objective is optimized for the remaining duration of training. 

\reminder{section flows you chose}

\reminder{talk about ammorization of flow parameters}
